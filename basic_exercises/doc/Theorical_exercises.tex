\documentclass[12pt]{article}
\usepackage[utf8]{inputenc}
\usepackage[acronym, toc]{glossaries}
\usepackage{amsmath,amsthm}
\usepackage{amssymb,amsfonts,latexsym,cancel}
\usepackage{graphicx}
\usepackage{float}
\usepackage{subfig}
\usepackage{ulem}
\usepackage[lmargin=3cm,rmargin=3cm,top=2.5cm,bottom=2.5cm]{geometry}
\usepackage{longtable}
\usepackage{ragged2e}
\usepackage{datetime}
\usepackage{tikz}
\usetikzlibrary{shapes, positioning}
\usepackage{parskip}
\usepackage{fancyhdr}
\usepackage{titling}
\pagestyle{fancy}
\setlength{\headheight}{15.71667pt}
\usepackage{hyperref}
\hypersetup{
    colorlinks,
    citecolor=black,
    filecolor=black,
    linkcolor=black,
    urlcolor=black
}

\fancyfoot{}
\renewcommand{\footrulewidth}{0.5pt}
\fancyfoot[R]{\thepage}

\title{ Basic Exercises - Theorical exercises % Título de la práctica
}
\author{ Sergio Marín Sánchez % Nombre del autor
}
\newdate{date}{8}{3}{2024} % Fecha

\begin{document}
\begin{titlepage}
    \centering
    \phantom{a}
    \vspace{2cm}
    {\includegraphics[scale=0.9]{ESCUDO_UPM.pdf}\par}
    \vspace{2cm}
    {\bfseries\LARGE Universidad Politécnica de Madrid \par}
    \vspace{1cm}
    {\scshape\Large Escuela Técnica Superior de Ingenieros Informáticos \par}
    \vspace{1cm}
    {\scshape\Huge \thetitle \par}
    \vfill
    {\large Author: \theauthor \par}
    {\large Email: \href{mailto:sergio.marins@alumnos.upm.es}{sergio.marins@alumnos.upm.es} \par } % email
    \vspace{0.2cm}
    {\large Date: \displaydate{date} \par}
\end{titlepage}

\thispagestyle{empty}
\tableofcontents
\clearpage

\setcounter{page}{1}

\section{Exercise 2b}

{
\bfseries 
Estimate the number of operations of operations to process a $n \times n$ input image using: $\delta_B(I)$ or $\delta_C\left(\delta_D \left( I \right) \right)$.
} 

\begin{center}
    \begin{tikzpicture}
        \node[] (B_label) {$B$ \quad $m \times m$};
        \node[draw, minimum width=2cm, minimum height=2cm] (B) [below = 0cm of B_label] {$\times$};
        
        \node[draw, minimum width=2cm, minimum height=0.5cm] (C) [right = 2 cm of B] {$\times$};
        \node[] (C_label) [above = 0cm of C] {$C$ \quad $1 \times m$};
        
        \node[draw, minimum width=0.5cm, minimum height=2cm] (D) [right = 1cm of C] {$\times$};
        \node[] (D_label) [above = 0cm of D] {$D$ \quad $1 \times m$};
    \end{tikzpicture}
\end{center}

First of all, it is defined the function $op(f)$ that counts the number of operations required to compute $f$. It is also defined that computing the maximum of two numbers $a, b \in \mathbb{Z}$ needs only 1 operation, thus, $op\left(\max\{a, b\}\right) = 1$.

Using these assumptions computing the maximum of $n$ numbers ($x_i$) will require $n - 1$ operations.
\begin{gather*}
    op(\underbrace{\max\{\cdots\max\{\max\{}_{\max{} \text{ is repeated } n - 1 \text{ times}}x_1, x_2\}, x_3\},\cdots,x_n\}) = n - 1
\end{gather*}

The number of operations required to compute $\delta_B(I)$ will be computed as the maximum among the structuring element $B$ that has $m^2$ elements, so it will require $m^2 - 1$ operations. This is performed in each pixel of the image, that is $n^2$ times.

The number of operations required to compute $\delta_C\left(\delta_D \left( I \right) \right)$ will be computed as the maximum among the structuring element $C$ that has $m$ elements, so it will require $m - 1$ operations, plus computing the maximum among the other structuring element $D$ that will also require $m - 1$ operations because it has $m$ elements too. This is performed in each pixel of the image, that is $n^2$ times.

In conclusion, 
    $$
        n^2(m^2 - 1) = op\left(\delta_B(I)\right) \geq op\left(\delta_C\left(\delta_D \left( I \right) \right)\right) = n^2(m - 1) + n^2(m - 1) = 2n^2(m - 1)
    $$.

\clearpage

\section{Exercise 9a}

{
\bfseries 
Proof the idempotence of the `\texttt{closing-opening}' alternated filter.
} 

To prove the idempotence of the closing-opening ($\gamma_B\varphi_B$) we first need to assume some statements:
\begin{itemize}
    \item Both $\gamma_B$ (opening) and $\varphi_B$ (closing) are increasing and idempotent.
    \item $\gamma_B$ (opening) is antiextensive.
    \item $\varphi_B$ (closing) is extensive.
\end{itemize}

\newtheorem{thm}{Theorem}

\phantom{a}

\begin{thm}
    Idempotence of the `closing-opening' alternated filter:
    $$
    \gamma_B\varphi_B\gamma_B\varphi_B(I) = \gamma_B\varphi_B(I)
    $$
    
    \begin{proof}
        The proof is done showing that both of the inequalities are satisfied.
        \begin{itemize}
            \item[]
            \noindent\fbox{%
                \parbox{0.3\textwidth}{%
                    $\gamma_B\varphi_B\gamma_B\varphi_B(I) \leq \gamma_B\varphi_B(I)$
                }%
            }
            \begin{equation}
            \begin{aligned}
                \gamma_B\varphi_B(I) &= \gamma_B\varphi_B(I) \Rightarrow \\
                \gamma_B\varphi_B\varphi_B\varphi_B(I) &= \gamma_B\varphi_B(I) \Rightarrow \\
                \gamma_B\varphi_B\gamma_B\varphi_B(I) &\leq \gamma_B\varphi_B(I)
            \end{aligned}
            \label{eq:leq}
            \end{equation}
            \item[]
            \noindent\fbox{%
                \parbox{0.3\textwidth}{%
                    $\gamma_B\varphi_B\gamma_B\varphi_B(I) \geq \gamma_B\varphi_B(I)$
                }%
            }
            \begin{equation}
            \begin{aligned}
                \gamma_B\varphi_B(I) &= \gamma_B\varphi_B(I) \Rightarrow \\
                \gamma_B\gamma_B\gamma_B\varphi_B(I) &= \gamma_B\varphi_B(I) \Rightarrow \\
                \gamma_B\varphi_B\gamma_B\varphi_B(I) &\geq \gamma_B\varphi_B(I)
            \end{aligned}
            \label{eq:geq}
            \end{equation}
        \end{itemize}
        By \eqref{eq:leq} and \eqref{eq:geq} it can be concluded that the equality $\gamma_B\varphi_B\gamma_B\varphi_B(I) = \gamma_B\varphi_B(I)$ holds. \\
    \end{proof}
\end{thm}

\end{document}
